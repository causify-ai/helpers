\documentclass[11pt, reqno]{amsart}
\usepackage{amsfonts, amssymb, amscd, amsrefs}
\usepackage{graphicx}
\usepackage{hyperref}
\usepackage{slashed}
\usepackage{fullpage}
% Prevent table repositioning.
\usepackage{float}
% For customizing list environments.
\usepackage{enumitem}
% For textcolor and cell colors
\usepackage[table]{xcolor}
\usepackage{colortbl}
% For blackboard bold `1`.
\usepackage{bbold}
% Settings for better figure placement
\usepackage{placeins}
\usepackage{afterpage}

\usepackage{multirow}
% Increase tolerance for figure placement
\setcounter{topnumber}{4}
\setcounter{bottomnumber}{4}
\setcounter{totalnumber}{6}
\renewcommand{\topfraction}{0.9}
\renewcommand{\bottomfraction}{0.9}
\renewcommand{\textfraction}{0.1}
\renewcommand{\floatpagefraction}{0.9}

\usepackage{booktabs}
\usepackage{tabularx}
\usepackage{longtable, array}

\usepackage{algorithm}
\usepackage{algpseudocode}

\usepackage{makecell}

%\usepackage{titlesec}

%\titleformat*{\section}{\LARGE\bfseries}
%\titleformat*{\subsection}{\Large\bfseries}
%\titleformat*{\subsubsection}{\large\bfseries}
%\titleformat*{\paragraph}{\large\bfseries}
%\titleformat*{\subparagraph}{\large\bfseries}

% If you need math, theorems, etc.
\usepackage{amsmath, amssymb, amsthm}
\usepackage{tikz}
\usetikzlibrary{
  positioning,
  calc,
  arrows.meta,
  decorations.pathreplacing,
  shapes.geometric,
  fit
}

% TikZ setup
\tikzset{
% Default arrow style
arrow/.style={->, >=stealth, thick},
% Global settings for all pictures
every picture/.style={ transform shape, scale=0.85, every node/.style={ scale=0.85, inner sep=2pt, outer sep=1pt } },
% Common node styles
box/.style={ rectangle, draw, minimum width=3cm, minimum height=1cm, text centered, align=center },
% Default text style
every text node part/.style={ font=\sffamily } }

\usepackage{pgf}
\usepackage[pdf]{graphviz}
\usepackage{uniquecounter}

% Code listings with minted.
% \usepackage{minted}

% define a light yellow
\definecolor{codebg}{RGB}{245, 245, 230} % very light yellow

% set defaults for python minted listings
% \setminted[python]{
%   bgcolor=codebg,
%   fontsize=\small,
% }

\usepackage[useregional]{datetime2}
\DTMlangsetup[en-US]{zone=eastern,mapzone}

% --- Theorem environments (acmart loads amsthm; this is safe) ---
\theoremstyle{plain}
\newtheorem{theorem}{Theorem}[section]
\newtheorem{lemma}[theorem]{Lemma}
\newtheorem{proposition}[theorem]{Proposition}
\newtheorem{corollary}[theorem]{Corollary}

\theoremstyle{definition}
\newtheorem{definition}[theorem]{Definition}

\theoremstyle{remark}
\newtheorem{remark}[theorem]{Remark}

\newtheorem{assumption}{Assumption}[section]

% Symbols
\newcommand{\cmark}{\textcolor{green!60!black}{\ensuremath{\checkmark}}}
\newcommand{\xmark}{\textcolor{red!70!black}{\ensuremath{\times}}}

% Qualitative strengths on a red→green scale
\newcommand{\Weak}{\textcolor{red!70!black}{Weak}}              % most red
\newcommand{\Limited}{\textcolor{red!50!orange}{Limited}}       % red-orange
\newcommand{\Partial}{\textcolor{orange!80!black}{Partial}}     % orange
\newcommand{\Moderate}{\textcolor{yellow!60!black}{Moderate}}   % yellow
\newcommand{\Strong}{\textcolor{green!60!black}{Strong}}        % green
\newcommand{\None}{\textcolor{gray}{None}}                      % neutral

% Custom data/code commands
\newcommand{\nan}{\textnormal{\texttt{nan}}}
\newcommand{\NaN}{\textnormal{\texttt{NaN}}}
% Dataframe.
\newcommand{\df}{\textnormal{\texttt{df}}}
\newcommand{\dtype}{\textnormal{\texttt{dtype}}}
%
% Table icons.
\newcommand{\trm}{\textcolor{red}{-}}
\newcommand{\tyo}{\textcolor{yellow}{o}}
\newcommand{\tgp}{\textcolor{green}{+}}

\input{./helpers_root/dev_scripts_helpers/documentation/latex_abbrevs.sty}

% Python style for highlighting
\usepackage{listings}

\newcommand{\pythonstyle}{\lstset{ language=Python, basicstyle=\ttm, morekeywords={self}, % Add keywords here
keywordstyle=\ttb
\color{deepblue}
, emph={MyClass,__init__}, % Custom highlighting
emphstyle=\ttb
\color{deepred}
, % Custom highlighting style
stringstyle=
\color{deepgreen}
, frame=tb, % Any extra options here
showstringspaces=false }}

% Python environment.
\lstnewenvironment{python}[1][]{ \pythonstyle \lstset{#1} }{}

% Python for external files.
\newcommand{\pythonexternal}[2][]{{ \pythonstyle \lstinputlisting[#1]{#2}}}

% Python for inline.
\newcommand{\pythoninline}[1]{{\pythonstyle\lstinline!#1!}}

\usepackage{listings}
\usepackage{xcolor}

\definecolor{codegreen}{rgb}{0,0.6,0}
\definecolor{codegray}{rgb}{0.5,0.5,0.5}
\definecolor{codepurple}{rgb}{0.58,0,0.82}
\definecolor{backcolour}{rgb}{0.95,0.95,0.92}
\definecolor{redbackground}{rgb}{1,0.9,0.9}
\definecolor{greenbackground}{rgb}{0.9,1,0.9}

\lstdefinestyle{mystyle}{ backgroundcolor=
\color{backcolour}
, commentstyle=
\color{codegreen}
, keywordstyle=
\color{magenta}
, numberstyle=\tiny
\color{codegray}
, stringstyle=
\color{codepurple}
, basicstyle=\ttfamily\footnotesize, breakatwhitespace=false, breaklines=true, captionpos=b,
keepspaces=true, numbers=left, numbersep=5pt, showspaces=false, showstringspaces=false,
showtabs=false, tabsize=2 }

\lstdefinestyle{redstyle}{ backgroundcolor=
\color{redbackground}
, commentstyle=
\color{codegreen}
, keywordstyle=
\color{magenta}
, numberstyle=\tiny
\color{codegray}
, stringstyle=
\color{codepurple}
, basicstyle=\ttfamily\small, breakatwhitespace=false, breaklines=true, captionpos=b,
keepspaces=true, showspaces=false, showstringspaces=false,
showtabs=false, tabsize=2 }

\lstdefinestyle{greenstyle}{ backgroundcolor=
\color{greenbackground}
, commentstyle=
\color{codegreen}
, keywordstyle=
\color{magenta}
, numberstyle=\tiny
\color{codegray}
, stringstyle=
\color{codepurple}
, basicstyle=\ttfamily\small, breakatwhitespace=false, breaklines=true, captionpos=b,
keepspaces=true, showspaces=false, showstringspaces=false,
showtabs=false, tabsize=2 }

\lstdefinestyle{codebgstyle}{ backgroundcolor=
\color{codebg}
, commentstyle=
\color{codegreen}
, keywordstyle=
\color{magenta}
, numberstyle=\tiny
\color{codegray}
, stringstyle=
\color{codepurple}
, basicstyle=\ttfamily\small, breakatwhitespace=false, breaklines=true, captionpos=b,
keepspaces=true, numbers=left, numbersep=5pt, showspaces=false, showstringspaces=false,
showtabs=false, tabsize=2 }

\lstset{style=mystyle}

% end python

%

\providecommand{\tightlist}{%
\setlength{\itemsep}{0pt}
\setlength{\parskip}{0pt}}

% Use this if using `\contrib[]{...}`.
%\makeatletter\let\@wraptoccontribs\wraptoccontribs\makeatother
% https://tex.stackexchange.com/questions/418547/equal-contribution-using-thanks-with-llncs-class#418563
\makeatletter
\newcommand{\printfnsymbol}[1]{%
\textsuperscript{\@fnsymbol{#1}}%
}
\makeatother
